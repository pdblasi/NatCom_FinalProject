% !TEX root = FinalProject.tex

\chapter{Background and Literature Review}

Our basic concept was to make a evolution simulation game. There are many examples of simulation games, many of them quite old. To cover examples of many of our observations when developing our project, we're going to look at three specific games: SimAnt, Spore, and Pandemic.

\section{SimAnt}
SimAnt is older than either member of the team being released in 1991. It was released for various systems and the name pretty much describes the gameplay. You can control many of the actions of a colony of ants. You can collect food and fight enemies yourself or summon nearby ants to assist you.

One of the main ways this game is similar to ours is in the automated gameplay. Many of patterns we see in our game are present in SimAnt when you let it go, namely, automated action does better than the player could ever hope to do.


\section{Spore}
Spore ties a little more directly into our genre because it is a game based on evolving a specific species and improving it through mutation and (to some extent) selection. This game is far more recent having been released in 2008. Setting aside the light-year gaps in graphics quality, there are two main differences between Spore and our game.

The first difference is the amount of control that the player has over the species evolution. In Spore, the player directly chooses mutations to improve their creature, while our game is guided by the user but essentially still created by random processes.

The second difference is in the types of mutations that occur in Spore. In Spore, most of the mutations are physically based. In our game, most of the mutations are behavioral. It is interesting to note the overlap between the games though. The most prominent is predator vs. prey mentality.


\section{Pandemic}
Pandemic is not a video game. It is actually a board game. Many similarities between board games and video games exist, and in fact many games have been adapted to both platforms. The similarities between our video game and the Pandemic board game lie in the random nature of the enemies.

In Pandemic, the players are all working cooperatively. This is not the case in our game, but many similarities can still be drawn between the enemies. The enemies in both games are randomly changed, though they are goal oriented. The trick to both is to be adaptable and know when to adjust the proper parameters. In the case of Pandemic, who has certain resources and where do they reside, and in our game focus points.


\section{Summary}
These games are similar in concept to ours, and there are many other out there as well, but ours has a unique flair of being very strongly guided by evolution as an algorithm. We will go more into detail on this in the next section.

